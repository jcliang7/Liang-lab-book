\labday{Wednesday, 27 September 2017}

\experiment{解決每次重新開啟終端機,需要重新設定\$GOPATH的問題。}


\begin{enumerate}
\item 問題原因

因為使用gvm作為go語言的版本控管器,選擇go版本後,會去抓g01.9這個檔案,裡面的環境變數是gvm預設的,所以每次都要重新設定。

\item 解決辦法

自訂一個環境變數,選擇go版本之後,指定環境變數檔。
	\begin{itemize}
	\item 選擇go版本,我選擇go1.9
	
	\colorbox{mygray}{\$ gvm use go1.9}

	\item 在此版本下加入自訂環境變數”test”
	
	\colorbox{mygray}{\$ gvm pkgset create test}
	\item 進到此資料夾「~./gvm/environments/」,找到「go1.9@test.txt」檔。修改此檔案,在檔案最後面加上以下兩行。
	
	\colorbox{mygray}{export GOPATH=”/home/user/goroot”}
	
	\colorbox{mygray}{export PATH=”\$GOPATH/bin:\$PATH”}
	\item 在終端機下指令,指定使用test環境變數
	
	\colorbox{mygray}{\$ gvm pkgset use test}
	\end{itemize}
\end{enumerate}

