\labday{Friday, 25 November 2011}

\experiment{example}

\lipsum[1]

%-----------------------------------------

\experiment{example2} % Multiple experiments can be included in a single day, this allows you to segment what was done each day into separate categories

\begin{figure}[H] % Example of including images
\begin{center}
\includegraphics[width=0.5\linewidth]{example_figure}
\end{center}
\caption{Example figure.}
\label{fig:example_figure}
\end{figure}

%-----------------------------------------

\experiment{example3}

\lipsum[3-5]

%----------------------------------------------------------------------------------------

\labday{Friday, 26 March 2010}

\experiment{table}

\begin{table}[H]
\begin{tabular}{l l l}
\toprule
\textbf{Groups} & \textbf{Treatment X} & \textbf{Treatment Y} \\
\toprule
1 & 0.2 & 0.8\\
2 & 0.17 & 0.7\\
3 & 0.24 & 0.75\\
4 & 0.68 & 0.3\\
\bottomrule
\end{tabular}
\caption{The effects of treatments X and Y on the four groups studied.}
\label{tab:treatments_xy}
\end{table}

Table \ref{tab:treatments_xy} shows that groups 1-3 reacted similarly to the two treatments but group 4 showed a reversed reaction.

%----------------------------------------------------------------------------------------

\labday{Saturday, 27 March 2010}

\experiment{Bulleted list example} % You don't need to make a \newexperiment if you only plan on referencing it once

This is a bulleted list:

\begin{itemize}
\item Item 1
\item Item 2
\item \ldots and so on
\end{itemize}

%-----------------------------------------

\experiment{example}

\lipsum[6]

%-----------------------------------------

\experiment{example2}

\lipsum[7]
%-----------------------------------------

%----------------------------------------------------------------------------------------
%	FORMULAE AND MEDIA RECIPES
%----------------------------------------------------------------------------------------

\labday{} % We don't want a date here so we make the labday blank

\begin{center}
\HRule \\[0.4cm]
{\huge \textbf{Formulae and Media Recipes}}\\[0.4cm] % Heading
\HRule \\[1.5cm]
\end{center}

%----------------------------------------------------------------------------------------
%	MEDIA RECIPES
%----------------------------------------------------------------------------------------

\newpage

\huge \textbf{Media} \\ \\

\normalsize \textbf{Media 1}\\
\begin{table}[H]
\begin{tabular}{l l l}
\toprule
\textbf{Compound} & \textbf{1L} & \textbf{0.5L}\\
\toprule
Compound 1 & 10g & 5g\\
Compound 2 & 20g & 10g\\
\bottomrule
\end{tabular}
\caption{Ingredients in Media 1.}
\label{tab:med1}
\end{table}

%-----------------------------------------

%\textbf{Media 2}\\ \\

%Description

%----------------------------------------------------------------------------------------
%	FORMULAE
%----------------------------------------------------------------------------------------

\newpage

\huge \textbf{Formulae} \\ \\

\normalsize \textbf{Formula 1 - Pythagorean theorem}\\ \\
$a^2 + b^2 = c^2$\\ \\

%-----------------------------------------

%\textbf{Formula X - Description}\\ \\

%Formula